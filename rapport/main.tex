\documentclass[a4paper, 11pt]{article}

\usepackage[utf8]{inputenc}
\usepackage[T1]{fontenc}
\usepackage[french]{babel}
\usepackage{graphicx}
\usepackage{amsmath}
\usepackage{amssymb}
\usepackage{hyperref}
\usepackage{listings}
\usepackage{color}
\usepackage{tcolorbox}

\definecolor{lightgray}{rgb}{0.98,0.98,0.98}
\definecolor{mygreen}{rgb}{0,0.6,0}
\definecolor{mygray}{rgb}{0.5,0.5,0.5}
\definecolor{mymauve}{rgb}{0.58,0,0.82}

\lstset{ 
  backgroundcolor=\color{lightgray},
  frame=single,
  rulecolor=\color{black},
  tabsize=2,
  commentstyle=\color{mygreen},
  stringstyle=\color{mymauve},
  keywordstyle=\color{blue},
}

\newtcbox{\mdbox}{on line,boxrule=0pt,boxsep=0pt,colback=lightgray,top=5pt,bottom=5pt,left=5pt,right=5pt,arc=1pt,fontupper=\ttfamily}

\def\siecle#1{\textsc{\romannumeral #1}\textsuperscript{e}~siècle}

\pagestyle{headings}

\title{Rapport de projet \og Programmation Impérative\fg \\ Le Morpion}
\author{C. DEFRETIERE}
\date{\today}

\begin{document}

\maketitle

\begin{abstract}
  Dans le cadre de l'unité d'enseignement \og Programmation Impérative\fg , ce rapport traitera du jeu nommé \og Morpion\fg et surtout de sa conception. Le but est d'expliciter l'implémentation de ce jeu réalisée en C, et de commenter la progression de ce projet.
\\

Bonne lecture
\end{abstract}

\newpage

\tableofcontents

\newpage

\section{Mise en place}

Ce programme est écrit en C respectant la norme ANSI, il est fortement recommandé d'utiliser un système linux.

\subsection{Compilation}
Un fichier "makefile" est fourni avec le projet, 

\mdbox{make}


\end{document}
